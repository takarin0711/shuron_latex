\documentclass[11pt,a4j, titlepage]{jarticle} %titlepageで表紙のページ番号をなくす
\usepackage[dvipdfmx,dvips]{graphicx}
\usepackage{otf}
\usepackage{amsmath,amssymb}
\usepackage{ascmac,here,txfonts}
\usepackage{listings,jlisting}
\usepackage{color}
\usepackage{epsfig}
\usepackage{fancybox}
\usepackage{epstopdf}
\usepackage{bm}
\usepackage{cases}
\usepackage{comment}
\usepackage{setspace}
\usepackage{subcaption}
\usepackage{array}
%%% jdummy.def
%
\DeclareRelationFont{JY1}{mc}{it}{}{OT1}{cmr}{it}{}
\DeclareRelationFont{JT1}{mc}{it}{}{OT1}{cmr}{it}{}
\DeclareFontShape{JY1}{mc}{m}{it}{<5> <6> <7> <8> <9> <10> sgen*min
    <10.95><12><14.4><17.28><20.74><24.88> min10
    <-> min10}{}
\DeclareFontShape{JT1}{mc}{m}{it}{<5> <6> <7> <8> <9> <10> sgen*tmin
    <10.95><12><14.4><17.28><20.74><24.88> tmin10
    <-> tmin10}{}
\DeclareRelationFont{JY1}{mc}{sl}{}{OT1}{cmr}{sl}{}
\DeclareRelationFont{JT1}{mc}{sl}{}{OT1}{cmr}{sl}{}
\DeclareFontShape{JY1}{mc}{m}{sl}{<5> <6> <7> <8> <9> <10> sgen*min
    <10.95><12><14.4><17.28><20.74><24.88> min10
    <-> min10}{}
\DeclareFontShape{JT1}{mc}{m}{sl}{<5> <6> <7> <8> <9> <10> sgen*tmin
    <10.95><12><14.4><17.28><20.74><24.88> tmin10
    <-> tmin10}{}
\DeclareRelationFont{JY1}{mc}{sc}{}{OT1}{cmr}{sc}{}
\DeclareRelationFont{JT1}{mc}{sc}{}{OT1}{cmr}{sc}{}
\DeclareFontShape{JY1}{mc}{m}{sc}{<5> <6> <7> <8> <9> <10> sgen*min
    <10.95><12><14.4><17.28><20.74><24.88> min10
    <-> min10}{}
\DeclareFontShape{JT1}{mc}{m}{sc}{<5> <6> <7> <8> <9> <10> sgen*tmin
    <10.95><12><14.4><17.28><20.74><24.88> tmin10
    <-> tmin10}{}
\DeclareRelationFont{JY1}{gt}{it}{}{OT1}{cmbx}{it}{}
\DeclareRelationFont{JT1}{gt}{it}{}{OT1}{cmbx}{it}{}
\DeclareFontShape{JY1}{mc}{bx}{it}{<5> <6> <7> <8> <9> <10> sgen*goth
    <10.95><12><14.4><17.28><20.74><24.88> goth10
    <-> goth10}{}
\DeclareFontShape{JT1}{mc}{bx}{it}{<5> <6> <7> <8> <9> <10> sgen*tgoth
    <10.95><12><14.4><17.28><20.74><24.88> tgoth10
    <-> tgoth10}{}
\DeclareRelationFont{JY1}{gt}{sl}{}{OT1}{cmbx}{sl}{}
\DeclareRelationFont{JT1}{gt}{sl}{}{OT1}{cmbx}{sl}{}
\DeclareFontShape{JY1}{mc}{bx}{sl}{<5> <6> <7> <8> <9> <10> sgen*goth
    <10.95><12><14.4><17.28><20.74><24.88> goth10
    <-> goth10}{}
\DeclareFontShape{JT1}{mc}{bx}{sl}{<5> <6> <7> <8> <9> <10> sgen*tgoth
    <10.95><12><14.4><17.28><20.74><24.88> tgoth10
    <-> tgoth10}{}
\DeclareRelationFont{JY1}{gt}{sc}{}{OT1}{cmbx}{sc}{}
\DeclareRelationFont{JT1}{gt}{sc}{}{OT1}{cmbx}{sc}{}
\DeclareFontShape{JY1}{mc}{bx}{sc}{<5> <6> <7> <8> <9> <10> sgen*goth
    <10.95><12><14.4><17.28><20.74><24.88> goth10
    <-> goth10}{}
\DeclareFontShape{JT1}{mc}{bx}{sc}{<5> <6> <7> <8> <9> <10> sgen*tgoth
    <10.95><12><14.4><17.28><20.74><24.88> tgoth10
    <-> tgoth10}{}
\DeclareRelationFont{JY1}{gt}{it}{}{OT1}{cmr}{it}{}
\DeclareRelationFont{JT1}{gt}{it}{}{OT1}{cmr}{it}{}
\DeclareFontShape{JY1}{gt}{m}{it}{<5> <6> <7> <8> <9> <10> sgen*goth
    <10.95><12><14.4><17.28><20.74><24.88> goth10
    <-> goth10}{}
\DeclareFontShape{JT1}{gt}{m}{it}{<5> <6> <7> <8> <9> <10> sgen*tgoth
    <10.95><12><14.4><17.28><20.74><24.88> tgoth10
    <-> tgoth10}{}
\DeclareFontShape{JY1}{mc}{m}{sc}{<5> <6> <7> <8> <9> <10> sgen*min
    <10.95><12><14.4><17.28><20.74><24.88> min10
    <-> min10}{}
\DeclareFontShape{JT1}{mc}{m}{sc}{<5> <6> <7> <8> <9> <10> sgen*tmin
    <10.95><12><14.4><17.28><20.74><24.88> tmin10
    <-> tmin10}{}
\endinput
%%%% end of jdummy.def


\definecolor{dkgreen}{rgb}{0,0.6,0} 
\definecolor{gray}{rgb}{0.5,0.5,0.5}
\definecolor{mauve}{rgb}{0.58,0,0.82}
\setlength{\oddsidemargin}{0mm}
\setlength{\textwidth}{170mm} 
\setlength{\topmargin}{-5mm}
\setlength{\textheight}{240mm}
\setlength{\columnsep}{8mm}
\setlength{\oddsidemargin}{-1.04cm}

\begin{document}

%タイトル
\begin{titlepage}
	\begin{center}
		\vspace{8ex}
		{\Large \bf 平成29年度修士論文}
		\vspace{3ex}\\
		\rule{\hsize}{2mm}
		\vspace{1mm}\\ 
		{\LARGE \bf 複数人で使用可能な3Dアイデアノートシステムの提案と実装} 
		\vspace{6mm}\\ 
		\rule{\hsize}{2mm} 
		\vspace{2.5cm} \\ 
		{\Large 電気通信大学大学院 情報理工学研究科 \\ 
		情報学専攻 メディア情報学プログラム} 
		\vspace{2ex} \\ 
		\renewcommand{\thefootnote}{\fnsymbol{footnote}} 
		{\Large 田 野 研 究 室} 
		\vspace{3ex} \\ 
		{\Large 指導教員 : 田 野 \ 俊 一 ({\em Tano Shun'ichi})} 
		\vspace{3ex} \\
		{\Large 学籍番号 : 1630012 \ / \ 猪 膝 \ 孝 之 ({\em Inohiza Takayuki})} 
		\vspace{5ex} \\ 
		{\Large 提出日 : 平成 30 年 1 月 29 日 ( 火 )} 
		\vspace{-5ex} \\ 
		\begin{verbatim} 
		\end{verbatim} 
	\end{center} 
\end{titlepage}

%概要
\begin{abstract}
	\ \ \ 本研究では、スマートフォンの内蔵カメラを用いて指のジェスチャを認識して操作をする方法について提案、実装し、有用性について評価した。

	近年スマートフォンが普及してきており、タッチ操作が可能になった事によって、ユーザーは従来の携帯電話よりも直感的な操作を行う事が出来るようになった。加えて、スマートフォンは以前よりも画面表示領域が大きくなっている。しかし、ユーザーがコンテンツを視認する際に手や指が邪魔になり、スマートフォンの表示領域の広さを最大限生かせなくなる。

	そこで本研究はスマートフォンの内蔵カメラを利用して指のジェスチャを認識することによって、表示領域の広さを生かすような操作方法を提案し、実装した。また、独自のテスト用ページを作成し、被験者のブラウジング実験により有用性について評価した。

	実験結果よりFingCVのほうがセッション数を増やせば増やすほどタスク完了時間が低下することがわかった。操作に慣れない人も多く、スクロールやズームの精度がそこまで良くなかったため、被験者8人中良いと思った人が1人、普通と思った人が4人、悪いと思った人が3人とFingCVの総合的な評価としてはあまり良くなかった。個人差もかなりあったように見られた。画面の表示領域を最大限生かすという点は、FingCVの場合スクロールやズームをしている際、画面全体の把握がしやすかったどうかという設問で、とても把握しやすいと思った人が2人、把握しやすいと思った人が3人と回答しており、改善されたという意見が多かったことがわかった。しかし、カメラのプレビュー画面に集中しすぎて画面全体が見づらくなってしまうという問題点も挙がった。

\end{abstract}

%目次
\tableofcontents
\newpage
\listoffigures
\newpage
\listoftables
\newpage
\section{はじめに}
\subsection{研究背景}


\subsection{研究目的}
本研究は、どの端末でも付属している内蔵カメラを用いて画面の表示領域を狭める事なく、スマートフォンを操作するような手法を提案、実装を行った。その後、評価実験を行い、有用性について評価を行った。

\newpage
\section{研究背景と従来研究}
\subsection{はじめに}
この章では、画面の表示領域を最大限生かすことを目的とした既存の研究についてそれぞれ特徴や問題点を述べる。

\subsection{背面から端末を操作}


\subsection{内蔵カメラを用いて端末を操作}


\subsection{おわりに}
本章では、画面の表示領域を最大限生かすことを目的とした既存の研究について述べた。次章では、本研究で作成したシステム、FingCVの概要について述べる。

\newpage
\section{本研究のコンセプト}
\subsection{はじめに}
この章では、本研究で作成したシステム、FingCVについて詳しく述べる。最初に、システムの主な仕様、操作方法について説明をし、その次に実装したそれぞれの機能について述べる。
\subsection{主な仕様、操作方法}



\subsection{スワイプ操作}


\subsection{ピンチイン、ピンチアウト操作}


\subsection{閲覧履歴の移動操作}


\subsection{おわりに}
本章では本研究で作成したシステム、FingCVについての主な仕様、操作方法について説明をし、その後に実装したそれぞれの機能について述べた。次章では、FingCVの実装環境や実装方法について詳しく述べる。

\newpage
\section{システム設計}
\subsection{はじめに}
この章では最初に本研究で作成したシステム、FingCVについての実装環境について説明をする。その次に指認識の方法について詳しく説明した後にブラウザの操作処理について詳しく述べる。

\subsection{実装環境}


\subsection{指認識}


\subsubsection{肌色領域の抽出}


\subsubsection{手の検出}


\subsubsection{指先の検出}


\subsection{ブラウザの操作処理}


\subsubsection{操作モードの切替}


\subsubsection{指先のジェスチャ検出}


\subsection{おわりに}
本章では本研究で作成したシステム、FingCVについての実装環境について説明をし、指認識、ブラウザの操作処理の方法について詳しく説明をした。次章では評価実験について述べる。

\newpage
\section{プロトタイプシステムの実装}
\subsection{はじめに}
この章では本研究で作成したシステム、FingCVについての評価実験を行う。最初に実験目的について述べ、その後に実験内容について詳しく説明をする。

\subsection{実験目的}
本実験により、FingCVを使用することによってスマートフォンの表示領域の広さを最大限生かすことができたかどうかについて評価をする。被験者には実験後、アンケートに回答してもらいFingCVの使用感についても評価をする。

\subsection{被験者情報}


\subsection{実験内容}


\subsection{アンケート内容}

\subsection{おわりに}
本章では本実験の目的、実験内容、アンケート内容について詳しく述べた。次章では実験で得られたタスク完了時間の結果やアンケート結果について詳しく述べる。

\newpage
\section{プロトタイプの評価実験}
\subsection{はじめに}
この章では本実験の測定結果やアンケート結果について詳しく述べる。それぞれの考察についても述べる。


\subsection{練習のべき乗則}


\subsection{測定結果}


\subsection{アンケート結果}
ここではアンケートの結果を示す。

\subsubsection{スクロールの精度}


\subsubsection{ズームの精度}


\subsubsection{疲労感の有無}



\subsubsection{スクロールやズームをしている際の画面全体の把握のしやすさ(タッチの場合)}



\subsubsection{スクロールやズームをしている際の画面全体の把握のしやすさ(FingCVの場合)}



\subsubsection{タッチ操作とFingCVの比較}


\subsubsection{総合的なFingCVの評価}


\subsubsection{追加してほしい機能}


\subsubsection{その他}


\subsection{おわりに}
本章では実験結果、アンケート結果について述べ、これらの結果についての考察を述べた。次章は本論文のまとめを述べ、今回の研究の改善すべき点や今後の展望について詳しく述べる。

\newpage
\section{実験結果と考察}
\subsection{まとめ}


\subsection{改善すべき点}


\subsection{展望}

\newpage
\section{おわりに}


\section*{謝辞}
ご指導を頂いた赤池先生、角田先生並びに、実験の被験者やその他の面でも日々ご助力を頂いた研究室の皆様方に厚くお礼を申し上げます。

\newpage
\begin{thebibliography}{数字}
  \bibitem{wigdor} Daniel Wigdor, Clifton Forlines, Patrick Baudisch, John Brnwell, Chia Shen: LucidTouch:A See-through Mobile Device, UIST’07 Proceedings of the 20th annual ACM on symposium on User interface software and technology, p.269-278 (2007)
  \bibitem{kobayashi} 小林 茂, 鈴木 宣也, 赤羽 亨, 蛭田 直, 近藤 崇司, 伊豆 裕一, 米山 貴久, 横内 恭人: 裏タッチインターフェース 画面の背面から操作するインターフェースの提案, インタラクション論文集2010, SA24 (2010)
  \bibitem{takeda1} 竹田 智, 岩田 満:内蔵カメラを用いたスマートフォン操作手法の提案, 平成24年度電子情報通信学会東京支部学生会研究発表会論文集, p.205 (2013)
  \bibitem{takeda2} 竹田 智, 岩田 満: 内蔵カメラを用いたジェスチャによるスマートフォン操作手法の検討, 情報処理学会第76回全国大会論文集, 4-25,26 (2014)
  \bibitem{okada} 岡田浩臣, 星野孝総: HMDを用いた仮想ガジェットの開発, ViEWビジョン技術の実利用ワークショップ講演論文集2012, ROMBUNNO.IS2-D3(2012)
  \bibitem{subway} Subway Map Visualization jQuery Plugin, <http://kalyani.com/2010/10/subway-map-visualization-jquery-plugin/>
  \bibitem{rosenzu} canvasを使って地下鉄の路線図を描けるjQueryプラグインで東京の地下鉄の路線図を描いてみた, <http://webdrawer.net/javascript/subwaycanvas.html>
\end{thebibliography}

\newpage
\section*{付録 本実験アンケート}
本実験後に回答してもらった各被験者のアンケート用紙を添付する。

\end{document}